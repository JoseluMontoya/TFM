\documentclass{article}

% %Basics
\usepackage{inputenc}
\usepackage[T1]{fontenc}
\usepackage{float}
\usepackage{enumitem}
% \usepackage[spanish, es-noquoting]{babel}
\usepackage[]{babel}

% % Visual flare and structure
\usepackage{tikz}
\usepackage{circuitikz}
\usepackage{graphicx}
\usepackage[a4paper]{geometry}
\usepackage[pagestyles, explicit]{titlesec}
\usepackage{caption}
\usepackage{subcaption}
\usepackage{soul}
% \usepackage{multicol}
\usepackage{abstract}
\usepackage{appendix}
\usepackage{lipsum}
\usepackage{svg}

% % Math
\usepackage{bm}
\usepackage{amsmath}
\numberwithin{equation}{section}
\usepackage[makeroom]{cancel}
\usepackage{empheq}
\usepackage{physics}
\usepackage{siunitx}

\AtBeginDocument{\RenewCommandCopy\qty\SI}
\ExplSyntaxOn
\msg_redirect_name:nnn { siunitx } { physics-pkg } { none }
\ExplSyntaxOff

% % Links
\usepackage[linktocpage]{hyperref}
\definecolor{cerulean}{rgb}{0.0, 0.48, 0.65}
\hypersetup{
    colorlinks=true,
    linkcolor=cerulean,
    filecolor=magenta,
    urlcolor=cyan,
    citecolor=magenta,
    % pdftitle={Overleaf Example},
    % pdfpagemode=FullScreen,
    }

% % Subfiles
\usepackage{subfiles}

% % Tikz libraries
\usetikzlibrary{babel}
\usetikzlibrary{calc}
\usetikzlibrary{math}
\usetikzlibrary{backgrounds}

\renewcommand{\CancelColor}{\color{red}}

\parskip=1em

\newcommand{\ROn}{R_{\text{On}}}
\newcommand{\ROff}{R_{\text{Off}}}
\newcommand{\LOn}{L_{\text{On}}}
\newcommand{\LOff}{L_{\text{Off}}}

\graphicspath{{./../code/img/}}

\title{
\textbf{SET readout using RF reflectometry and kinetic inductance nonlinearity}
}
\author{Jose Luis Montoya Agull\'o}
\date{}

\begin{document}
\newgeometry{top=2cm}
\maketitle
\begin{abstract}
    In this thesis we will make a proof of concept of a new type of charge sensor for quantum computation,
    using an SET, RF reflectometry and a kinetic inductance, which is the novel
    part of the sensor. First we will optimize the sensor analytically using a non-kinetic inductance,
    and then we will simulate the kinetic version to see if we can improve the fidelity.
    The results show that, for a non-kinetic sensor, the optimum tuning is done to
    the geometric mean of the two resistances we want to distinguish between, and
    that the use of a kinetic inductor with a proper calibration increases greatly
    the fidelity of the sensor, specially with parasitic resistances of the order
    of the smallest resistance.
\end{abstract}


\tableofcontents{}
\restoregeometry{}

\newpage
\section{Introduction}
\label{sec:Intro}
\subfile{sections/Introduction.tex}

\section{Aims and objectives}
\label{sec:Objectives}
\subfile{sections/Objectives.tex}

\newpage
\section{Theoretical background}
\label{sec:Theory}
\subfile{sections/Theory.tex}

\newpage
\section{Results}
\label{sec:Results}
As we said in section \ref{subsec:RFRef}, in RF reflectometry we use
the reflection coefficient \(\Gamma\) (\ref{eq:ReflecCoeff}) to measure an
impedance \(Z(\omega)\), and if said impedance can have two possible values,
the relevant parameter to determine the distinguishability of those measures
is the signal-to-noise ratio SNR (\ref{eq:SNR}).

Since \(Z(\omega)\) can't be modified (it defeats the purpose of measuring it),
and \(Z_{0}\) depends on the transmission line (it usually is
\(50\unit{\ohm}\)), the standard procedure is to add a matching network
(essentially, extra circuitry around \(Z(\omega)\)) and instead measure the
impedance of said network plus \(Z(\omega)\). The idea is that with the
matching network we bring the impedance of the ensemble closer to \(Z_{0}\),
increasing the variation of \(\Gamma\) with respect to the impedance we want
to measure, and with it improving our SNR via the contrast.

\begin{figure}
\centering
\begin{circuitikz}[scale=0.9]
    \draw (-0.5,0) to[short]
          (0,0) to[generic, l=\(Z_0\)]
          (1.5,0) to[capacitor, l=\(C_c\)]
          (3,0) to[short]
          (6,0);
    \draw (3,0) to[L, l=\(L\)]
          (3,-2);
    \draw (4.5,0) to[capacitor, l=\(C_p\)]
          (4.5,-2);
    \draw (6,0) to[vR, l=\({R = R_{SET} \parallel R_p}\)]
          (6,-2);
    \draw (3,-2) to[short]
          (6,-2) node[ground]{};
\end{circuitikz}
\caption{Topology of our measuring circuit. Since during most of our
analysis we will ignore \(R_{p}\), it is already couplet to \(R_{SET}\).}
\label{fig:RLC}
\end{figure}

In our case, the impedance to measure is the resistance of the SET (\(R_{SET}\)),
which has 2 states: \(\ROn^{SET} = 50\unit{\kilo\ohm}\)\footnote{I know that in
    section \ref{subsec:SET} we saw that it must be a lot greater than
\(51.6\unit{\kilo\ohm}\), but we're going to work in a worst case scenario} and
\(\ROff^{SET} = \infty\unit{\ohm}\) (zero current passes). For our matching network,
since \(R_{SET} \gg Z_{0}\) we will use the appropriate version of what is
known as a high pass L matching network. It consists of a coupling capacitance
\(C_{c}\) connected in series with \(Z_{0}\) and \(R_{SET}\), and an inductance
\(L\) in series with \(R_{SET}\). In figure \ref{fig:RLC} we can see this
arrangement with the addition of a parasitic capacitance and resistance,
\(C_{p}\) and \(R_{p}\), which are used to model the losses in the circuit.
Due to its shape, we will refer to it as the parallel RLC resonator, or the
resonator for short.

The objectives of this section are to first study our measuring circuit,
find the optimal values of its parameters, and see what that ideal
measurement would look like, while along the way keep doing simulations
to ensure that our results work. Next, we will check if with a kinetic inductor
the performance can be improved, and under what circumstances said improvement
occurs.

\subsection{The parallel RLC resonator}
\label{subsec:ParallelRLC}
\subfile{sections/RLC.tex}

\subsection{The parallel kinetic RLC resonator}
\label{subsec:KineticRLC}
\subfile{sections/KineticRLC.tex}

\newpage
\section{Conclusions}
\label{sec:Conclusions}
\subfile{sections/Conclusions.tex}

\section{Next steps and closing remarks} %Vistas de futuro: que hacer despues
\label{sec:NextSteps}
\subfile{sections/NextSteps.tex}

\nocite{zhaoPhysicsSuperconductingTravellingWave2022}
\bibliographystyle{plain}
\bibliography{../TFM}

% \newpage
% \appendix \appendixpage \addappheadtotoc
%
% \section{Cuentas}
% \subfile{sections/Ap_1}

\end{document}

