\documentclass[../main.tex]{subfiles}

\begin{document}
\subsection{The SET for charge sensing}

\hl{Hola Fernando, creo que he terminado de entender lo suficientemente bien
como para escribir esta seccion el Coulomb blockade y el SET. El problema
es que estoy escribiendo esto a las 18:06, asi que voy a escribirlo primero
a grandes rasgos para que me puedas decir si conceptualmente lo entiendo bien,
y una vez que eso este subido iré añadiendo las cosas de forma mas rigurosa.}

Let's imagine a conducting island on capacitance \(C\) and without excess charges.
Due to it's capacitance, some energy will be required to add a charge to this
island, and due to Coulomb repulsion each aditional charge will require even more
energy than the last. Since this aditional energy is also proportional to
\(1/C\), with a small enough island and with small enough thermal energy,
the energy spectrum of these charges will be discretized, blocking all charges
from entrance unless they have enough energy to arrive to the next empty
energy state. It's like the quatization of the energy spectra of the 1D infinite
square well, but from classical properties.

A SET is a conducting island connected to 2 voltages via tunnel junctions
and to a third one via a capacitor. The tunnel junctions are modeled via
a capacitor, to simulate the charge acumulation at the walls of the tunnel
juction, and a resistance in parallel to simulate the current flowing due to 
tunneling. This resistance must be high enough so that each tunneling event is
well defined, and the way to ensure that is to plug the energy required to
introduce an electron to the capacitor, and the RC time of the tunnel junction
(in other context the RC time of a circuit is the time it takes to charge a
capacitor through a resistance, but in this one would be the average time
it takes a charge to tunnel through the barrier). By doing this we get
that \(R \gg 4.1\unit{\kilo\ohm}\), but a more careful analysis with
tunneling rates[REFERENCE HERE] gives us a more restrictive condition
\(R \gg 25.8\unit{\kilo\ohm}\). The voltage connected via a capacitor (the
gate voltage) would increase and decrease the electrostatic energy of the
island, allowing us to move the next avariable energy state inside it. The
idea is to have a small enough voltage difference between the voltages connected
via tunnel barrier, such that we can add a gate voltage that will let a current
flow by encouraging a single electron transport from one voltage to the next
through the islan, but that a small variation of the gate voltage breaks this
electron transmission and cuts the conductance of the SET
(being right above the edges of the Coulomb diamonts). This charge sensing
can be turned into spin sensing by making the source of the gate voltage be
an electron (or the lack of one) in another quantum dot, and making the
existance (or lack of) an electron in said quantum dot dependent on it's spin.

\subsection{RF reflectometry}
Radio frequency reflectometry is a method to measure change in an impedance
connected to a transmission line via the reflection of a radio signal.

Usual lumped element treatment of AC circuits assumes that the size of the
circuit is small with respect to the wavelength of the voltage,
but with radio frequency voltages we can't do that. The main consequences for
us are that now voltage and current are dependent on how far along the circuit
you are due to how fast it changes with respect to the size of the circuit itself
(\(V(t), I(t) \rightarrow V(x, t), I(x, t)\)) and that the intrinsic
inductance and capacitance per unit length of a long connection cannot be
ignored. In such cases those connections are made with what is known as a
transmission line, which is a cable designed to minimize the radiation of
power via that inductance and capacitance, and includes a signal and a
ground connection in one package. One example of a transmission line would be a
coaxial cable, in which the central conductor is the signal and the outer jacket
ground.

\begin{figure}[t]
\centering
\begin{circuitikz}[]
    \newcommand{\lenghOfTransmissionLine}{6}
    \draw (0,0) node[ground]{};
    \draw (0,0) to[sV] ++(0,1.5) to[short] ++(1.5,0);
    \draw (1.5,0) to[capacitor] ++(0,1.5) to[L] ++(1.5,0) to[short] ++(0.5,0);
    \foreach \i in {2,...,\lenghOfTransmissionLine}{
        \draw (2*\i - 0.5,0) to[capacitor] ++(0,1.5) to[L] ++(1.5,0) to[short] ++(0.5,0);
    }
    \draw ({2*(1+\lenghOfTransmissionLine)-0.5},0) node[ground]{};
    \draw ({2*(1+\lenghOfTransmissionLine)-0.5},1.5) to[generic, l=\(Z\)] ++(0,-1.5);
    \draw (0,0) to[short] ++({2*(1+\lenghOfTransmissionLine)-0.5},0);
    \draw[dashed] (0.93,-0.15) rectangle ++(2,2);
    \draw [latexslim-latexslim] (2.93,-0.35) -- node[below] {\(\Delta x\)} ++(-2,0);
\end{circuitikz}
\caption{Lumped element model of a lossless transmission line connected to a
    generic impedance \(Z\). In order to model the relevant
    impedance per unit length of the transmission line at radio frequencies,
    we represent it via sections with inductors in series and capacitors
    in parallel. Each periodic section like the one inside the dashed rectangle
    represents a segment of length \(\Delta x\) of the transmission line.}
\label{fig:LumpedTransmisionLine}
\end{figure}

Even though a transmission line minimizes that radiation of power, it doesn't
erase it, and we need to take it into account in our calculations. To model this
inductance and capacitance per unit length (\(L_{l}\) and \(C_{l}\) respectively),
we will discretize it via a lumped element representation like in figure
\ref{fig:LumpedTransmisionLine}, ignoring the lossess by not including any
resistance  in our circuit.
Each pair inductor-capacitor will occur along a length \(\Delta x\) of the
transmission line, so their inductance and capacitance in that stretch will be
\(L = \Delta x \cdot L_{l}\) and \(C = \Delta x \cdot C_{l}\). Applying Kirchhoff
at a point \(x\) in the transmission line and the limit \(\Delta x \to 0\) gives
us the telegraph equations for a lossless transmission line

\begin{equation}
\begin{split}
\label{eq:TelegraphEq}
    \frac{\partial V}{\partial x} = - L_{l}\frac{\partial I}{\partial t}\\
    \frac{\partial I}{\partial x} = - C_{l}\frac{\partial V}{\partial t}\\
\end{split}
\end{equation}

Their solution is

\begin{align}
\begin{split}
    \label{eq:TelegraphSol}
    V(x,t) &= V_{+}(t - x/v_{p}) + V_{-}(t + x/v_{p})))\\
    I(x,t) &= \frac{1}{Z_{0}}(V_{+}(t - x/v_{p}) - V_{-}(t + x/v_{p}))
\end{split}
\end{align}

Where \(v_{p} = \frac{1}{\sqrt{L_{l}C_{l}}}\) is the phase velocity of the wave,
\(Z_{0} = \sqrt{\frac{L_{l}}{C_{l}}}\) is the characteristic impedance of the
line and \(V_{+}\) and \(V_{-}\) are generic functions that describe a right
and left traveling wave respectively. Since we'll be choosing our reference
frame such that our signal will be always traveling from left to right, the
appearance of \(V_{-}\) in our calculations will mean a reflection.

If we add a generic impedance \(Z\) at the end of the transmission line, we
add the boundary condition

\begin{equation}
\label{eq:RfBoundCond}
    \frac{V(x_{\text{End}},\omega)}{I(x_{\text{End}},\omega)} = Z(\omega)
\end{equation}

With \(V(x,\omega)\) and \(I(x,\omega)\) being the time Fourier transforms of
\(V(x,t)\) and \(I(x,t)\) respectively. Choosing \(x=x_{\text{End}}=0\) to
simplify and using the time Fourier transforms of expressions
\ref{eq:TelegraphEq}, we get

\begin{equation}
\label{eq:TelegraphEqFourierWithBound}
    \frac{V(0,\omega)}{I(0,\omega)} =
    Z_{0}\frac{V_{+}(\omega) + V_{-}(\omega)}
    {V_{+}(\omega) - V_{-}(\omega)} = Z(\omega)
\end{equation}

As we can see, the only way in which \(Z_{0} = Z(\omega)\) is if
\(V_{+,-}(\omega)=0\), or in other words, the only way to not get a reflection
is to match \(Z_{0}\) to \(Z(\omega)\). A useful parameter to define is the
reflection coefficient \(\Gamma = \frac{V_{-}(\omega)}{V_{+}(\omega)}\), and
with equality \ref{eq:TelegraphEqFourierWithBound} has the form

\begin{equation}
\label{eq:ReflecCoeff}
    \Gamma = \frac{Z(\omega) - Z_{0}}{Z(\omega) + Z_{0}}
\end{equation}

Through this is how we are going to measure \(Z(\omega)\), in our case
\(R_{\text{SET}}\), since there is a 1 to 1 mapping between \(\Gamma\) and
\(Z(\omega)\).

The quality of our measurement will be determined by the signal-to-noise ratio
(SNR) of the measurement

\begin{equation}
\label{eq:SNR}
\text{SNR} = |\Delta\Gamma|^2\frac{P_{0}}{P_{N}}
\end{equation}

With \(P_{0}\) and \(P_{N}\) the power of the signal and noise respectively and
\(\Delta\Gamma = \Gamma_{B} - \Gamma_{A}\) the difference in reflection
coefficients between the two states to measure. Its modulus is what we will
call the contrast and our objective will be first to optimize it by designing
a matching network\footnote{A circuit build around our impedance with the
objective of increasing the sensibility to its changes. We will talk more
about it in section \ref{sec:ParallelRLC}} for our SET, and then try to improve
it by adding a kinetic inductor, whose inductance changes with the current that
goes through it.

% A typical transmission line has a characteristic impedance of
% \(50\unit{\ohm}\), which is a lot greater than the lowest resistance of an
% SET, so a matching circuit around the SET will be needed in order to stay in
% an area in which

\subsection{Kinetic inductance and his nonlinearity}
Due to the high mobility of the charge carriers in a super conductor, a phenomenon
known as kinetic inductance emerges. This name comes from the fact that as opposed
to a usual inductor, which functions by storing energy in the magnetic field
generated by the charge carriers, it is stored as the kinetic energy of the
charge carriers themselves.

With this simple definition of the kinetic inductance and a little bit of
Drude and Ginzburg-Landau theory, we have all we need to derive the property
that interests us the most: it's nonlinearity.

The energy stored by an inductor of inductance \(L\) is

\begin{equation}
\label{eq:InductiveEnergy}
    E = \frac{1}{2}LI^2
\end{equation}

So, in the case of the kinetic inductance of length \(l\) and cross-section \(S\)

\begin{equation}
\label{eq:KineticInductiveEnergy}
    E_{k} = \frac{1}{2}L_{k}I^2 = \frac{1}{2}m(nlS)v^2
\end{equation}

With \(n\), \(m\) and \(v\) the volumetric density, the mass and the speed of
the charge carriers. By solving for \(L_{k}\) and defining the volumetric
current density as \(j=nqv\) with \(q\) the charge of the charge carriers
we arrive at the following expression

\begin{equation}
\label{eq:LkOfn}
    L_{k} = \frac{mlj^2}{nq^2Sj^2} = \frac{ml}{q^2S}\frac{1}{n}
\end{equation}

Now, using the Ginzburg-Landau expression for the volumetric density of supercharge
carriers

\begin{equation}
\label{eq:GLn}
    n(v) = |\Psi|^2 = \frac{1}{\beta}\left[|\alpha| - \frac{1}{2}m v^2\right]
\end{equation}

And doing a second order approximation of \(L_{k}\) at \(v \approx 0\), we arrive
at our desired expression

\begin{equation}
\label{eq:KineticNonLineality}
    L_{k} = L_{k0}\left(1 + \frac{j^2}{j_{*}^2} + \dots\right)
\end{equation}

With \(L_{k0} = \frac{ml}{q^2Sn(v=0)}\) and \(j_{*}^2 =
\frac{2q^2|\alpha|^3}{m\beta^2}\). If we compare \(j_{*}\)
to the critical current \(j_{c}\), which is the maximum current that
the system can withstand and can be calculated by obtaining the maximum of
\(j\) with respect to \(v\), we get that

\begin{equation}
\label{eq:NonLinearityConstant}
    j_{*} = \sqrt{\frac{27}{4}}j_{c}
\end{equation}

With this we can not only see that the kinetic inductance has a quadratic
dependence with the current, but that the sensibility of that dependence it's
given by the critical current of the material. And these properties are the
ones that we what to use to improve the reading quality for our qubits.

But, to read the qubits with RF reflectometry we send an AC pulse to the
resonator, and in that case our kinetic inductance would be varying constantly.
How can we use the kinetic inductor then?

The solution is to introduce a DC bias to the circuit, with an intensity much
greater than the maximum of the AC current, but still small enough to not break
superconductivity and to use an AC current much smaller than the critical current.
With it, we can have an inductor that changes inductance along with the resistance
of the SET, since the effects of the AC current can be ignored.

\begin{align}
\begin{split}
\label{eq:KineticNonLinealityACDC}
L_{k} &= L_{k0}\left(1 + \frac{(j_{AC} + j_{DC})^2}{j_{*}^2} + \dots\right)\\
      &= L_{k0}\left(1 + \frac{j_{DC}^2}{j_{*}^2} + \frac{j_{DC}j_{AC}}{j_{*}^2} + 
      \frac{j_{AC}^2}{j_{*}^2} + \dots\right)\\
      &= L_{k0}\left(1 + \frac{j_{DC}^2}{j_{*}^2} + (\frac{j_{DC}}{j_{*}} + 
      \frac{j_{AC}}{j_{*}})\frac{j_{AC}}{j_{*}} + \dots\right)\\
      &= L_{k0}\left(1 + \frac{j_{DC}^2}{j_{*}^2} + \dots\right)
      \text{ since \(j_{DC} < j_{c}\) and \(j_{AC} \ll j_{c}\)}
\end{split}
\end{align}


\end{document}
