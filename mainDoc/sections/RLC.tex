\documentclass[../main.tex]{subfiles}

\begin{document}
\subsubsection{Resonant frequency and effective impedance}
Our analysis begins with obtaining expressions for the resonant frequency and
the effective impedance of our resonator. It's easy to see that its impedance is
\begin{equation}
\label{eq:ImpParallel}
    Z(\omega) = \frac{1}{j \omega C_{p} + \frac{1}{j \omega L} + \frac{1}{R}}
        + \frac{1}{j \omega C_{c}}
\end{equation}
Which after a little massaging turns into
\begin{equation*}
\label{eq:ImpParallelBinomial}
    Z(\omega) = \frac{\omega^2 L^2 R}{R^2(1-\omega^2C_{p}L)^2 + \omega^2 L^2} +
        j \left(
            \frac{\omega L R^2 (1-\omega^2C_{p}L)}{R^2(1-\omega^2C_{p}L)^2 + \omega^2 L^2}
            - \frac{1}{\omega C_{c}}
          \right)
\end{equation*}

The resonant frequency \(\omega_{r}\) that makes \(\Im Z(\omega) = 0\) is
\begin{equation*}
\label{eq:ExactWr}
    \omega_{r}^2 = \frac{1}{L(C_{c} + C_{p})}
    \left(
        1 + \frac{C_{c}}{2C_{p}} - \frac{L}{2 R^2 C_{p}} \pm
        \sqrt{\left(1 + \frac{C_{c}}{2C_{p}} - \frac{L}{2 R^2 C_{p}}\right)^2
        - 1 - \frac{C_{c}}{C_{p}}}
    \right)
\end{equation*}

Choosing \(C_{c}\) and \(L\) such that
\(\frac{C_{c}}{C_{p}}, \frac{L}{R^2 C_{p}} \ll 1\), leaves us with the
approximate expression for the resonant frequency
\begin{equation}
\label{eq:Wr}
\omega_{r} \approx \frac{1}{\sqrt{L (C_{c} + C_{p})}}
\end{equation}

Finally, to obtain the effective impedance we use this expression in \(\Re Z\)
\begin{equation*}
\label{eq:ExactZeff}
    Z_{eff} = \Re Z(\omega_{r}) =
    \frac{\omega_{r}^2 L^2 R}{R^2(1-\omega_{r}^2C_{p}L)^2 + \omega_{r}^2 L^2}
    \approx \frac{L (C_{c} + C_{p})}{R C_{c}^2}
      \left(1 + \frac{L (C_{c} + C_{p})}{R^2 C_{c}^2}\right)^{-1}
\end{equation*}

And by, again, choosing \(L\) and \(C_{c}\) such that
\(\frac{L (C_{c} + C_{p})}{R^2 C_{c}^2} \ll 1\) we arrive to our expression for
the effective impedance
\begin{equation}
\label{eq:Zeff}
    Z_{\text{eff}} \approx \frac{L (C_{c} + C_{p})}{R C_{c}^2}
\end{equation}

During the optimization of the resonator we will be using quite a lot of
expressions obtained via approximations in non-approximated systems, only to do
more approximations with them. Due to this, it is really important to have a
clear picture of the regimes we are working in to ensure that our results work
in the state-of-the-art technology, and that is why after each result we are
going to recontextualize our approximations.

In this case, the approximations to obtain \(\omega_{r}\) are clear and
straight forward:

\begin{gather}
    \frac{C_{c}}{C_{p}} \ll 1 \label{eq:ApproxCcWr}\\
    \frac{L}{R^2 C_{p}} \ll 1 \label{eq:ApproxLWr}
\end{gather}

But the approximation for \(Z_{eff}\) needs a little bit of extra work. If we
multiply \((C_{c} / C_{p})^2\) in both sides, it turns into

\begin{equation}
\label{eq:ProtoZeffCond}
    \frac{L}{R^2 C_{p}} \left(1 + \frac{C_{c}}{C_{p}}\right) \ll
    \left(\frac{C_{c}}{C_{p}}\right)^2
\end{equation}

And since we used equation \ref{eq:Wr} to arrive here, it must hold
the approximation \ref{eq:ApproxCcWr}, turning the previous expression into

\begin{equation}
\label{eq:ApproximationZeff}
    \frac{L}{R^2 C_{p}} \ll \left(\frac{C_{c}}{C_{p}}\right)^2
\end{equation}

While approximations \ref{eq:ApproxCcWr} and \ref{eq:ApproxLWr} impose a
general condition in our degrees of freedom, approximation
\ref{eq:ApproximationZeff} imposes a relative condition between
the previous two.

For checking that our results are correct, we can graph the modulus \(\Gamma\)
as a function of the voltage frequency \(\omega\). With the parameters
listed in figure \ref{fig:ReflexCoeffAndPhase}, the 3 conditions for our
approximations are met, and with a resistance \(R = 50\unit{\kilo\ohm}\)
(so \(R_{p} = \infty\unit{\ohm}\)),
\(Z_{\text{eff}} \approx Z_{0}\) around \(\omega_{r} = 1.00654 \unit{\giga\hertz}\).
This means that \(|\Gamma|\) should dip to 0 quickly around \(\omega_{r}\),
which is exactly what we see.

\begin{figure}[t]
\centering
\begin{subfigure}[T]{.5\textwidth}
  \centering
  \includesvg[width=\linewidth]{RLCParallel/ReflecCoeff.svg}
\end{subfigure}%
\begin{subfigure}[T]{.5\textwidth}
  \centering
  \includesvg[width=\linewidth]{RLCParallel/ReflecPhase.svg}
\end{subfigure}
\caption{Modulus and phase of \(\Gamma\) in multiple configurations.
\(Z_{0}=50\unit{\ohm}\), \(C_{p} = 500\unit{\fF}\), \(C_{c} = 100\unit{\fF}\),
\(L = 41.67\unit{\nH}\).
The script generating this image can be found in the
companion Github repository with the name \texttt{ReflecGraph.py}.}
\label{fig:ReflexCoeffAndPhase}
\end{figure}

In addition to this configuration, we have also graphed the reflection coefficient
of the same resonator, but with variations in the resistance (\(R = 2\ROn\) and
\(R = \ROn/2\)) and a slight variation in the inductance (99\% of the previous
one). The variation of the resistance serves as an example of a behavior that
we will observe later: even though they are not the same distance in \(R\)
space from the resonant resistance, they are in \(\Gamma\) space, with directly
opposing positions in the complex plane as can be seen in the phase graph.
This is because, while in resonance, the distance in \(\Gamma\) space does
not depend on the distance in \(R\) space, it depends on the relative
distance, with the direction determined by which resistance (the resonant or
the perturbed one) is bigger. The variation in the inductance serves as another
way of creating distance between two values of \(\Gamma\), and our hope
is that it can work alongside a resistance change.


\subsubsection{Contrast and optimization}
\label{subsubsec:OptParallel}
With an expression for the effective impedance of the system in resonance and
an expression for the resonant frequency, we can begin the search for the
optimum parameters of the circuit. Or in other words, what combination
of parameters will yield the greatest contrast, and in turn the
greatest SNR.

Since the values of the parameters used on the circuit of figure
\ref{fig:ReflexCoeffAndPhase} are of the same order of magnitude as
the ones used in the lab for similar circuits\cite{ibbersonDispersivereadout},
it is safe to assume that the values of \(L\) and \(C_{c}\) will be of the
around the same size. With this in mind is easy to see that our approximation of
\(\omega_{r}\)
will be a lot more sensible to changes in \(L\) than to changes in \(C_{c}\),
and thus we will use \(C_{c}\) to optimize the contrast, while we will use \(L\)
to ensure that we stay in an acceptable frequency of operation.

We begin obtaining a workable expression of the contrast
by plugging \[\omega = \frac{1}{\sqrt{L(C_{c} + C_{p})}}\] into
\ref{eq:ImpParallel}:
\begin{align}
\label{eq:ImpParallelResonant}
\begin{split}
    Z(\omega) &= \frac{1}{j \omega C_{p} + \frac{1}{j \omega L} + \frac{1}{R}}
        + \frac{1}{j \omega C_{c}}\\
      &= \frac{  \omega R L}{R(1 - \omega^2 L C_{p}) + j \omega L}
        + \frac{1}{j \omega C_{c}}\\
      &= \frac{\frac{j R L}{\sqrt{L (C_{c} + C_{p})}}}{
          R\left(1 - \frac{\cancel{L} C_{p}}{\cancel{L}(C_{c} + C_{p})}\right)
            + \frac{j L}{\sqrt{L (C_{c} + C_{p})}}
            } + \frac{\sqrt{L (C_{c} + C_{p})}}{j C_{c}}\\
      &= \frac{j R L}{R\sqrt{L (C_{c} + C_{p})}
          \left(\frac{C_{c}}{C_{c} + C_{p}}\right) + j L}
          + \frac{\sqrt{L (C_{c} + C_{p})}}{j C_{c}}\\
      &= \frac{j R \cancel{L}}{R\frac{\cancel{L} \cancel{(C_{c} + C_{p})}}{
          \sqrt{L (C_{c} + C_{p})}}
          \left(\frac{C_{c}}{\cancel{C_{c} + C_{p}}}\right) + j \cancel{L}}
          + \frac{\sqrt{L (C_{c} + C_{p})}}{j C_{c}}\\
      &= \frac{j R}{RS + j} + \frac{1}{jS}
       = \frac{\cancel{-RS} + \cancel{RS} + j}{jRS^2 -S}
       = \frac{1}{RS^2 + jS}
       \text{ with } S = \frac{C_{c}}{\sqrt{L(C_{c} + C_{p})}}
\end{split}
\end{align}

Using this expression to obtain the reflection coefficient,
but using the admittance of the transmission line instead of the impedance
(\(Y_{0} = 1/Z_{0}\)) yields
\begin{align*}
\begin{split}
    \Gamma &= \frac{Z(\omega) - Z_{0}}{Z(\omega) + Z_{0}}
            = \frac{Y_{0} - 1/Z(\omega)}{Y_{0} + 1/Z(\omega)}\\
           &= \frac{2Y_{0}}{Y_{0} + 1/Z(\omega)} - 1 = \frac{2 Y_{0}}{RS^2 + Y_{0} + jS} - 1\\
           &= 2 Y_{0}\frac{RS^2 + Y_{0} - jS}{(RS^2 + Y_{0})^2 + S^2} - 1
\end{split}
\end{align*}

Since \(\omega \approx \omega_{r}\) then
\(\Im Z(\omega) \approx 0\) and by extension \(\Im \Gamma \approx 0\), so
\begin{equation*}
    \Gamma \approx 2 Y_{0}\frac{RS^2 + Y_{0}}{(RS^2 + Y_{0})^2 + S^2} - 1
\end{equation*}

Next, using the parameters utilized for figure \ref{fig:ReflexCoeffAndPhase}
to get a sense of the scale, it is safe to assume that the following
approximation is correct
\begin{equation}
\label{eq:ParallelContrAprox}
    (RS^2 + Y_{0})^2 \gg S^2
\end{equation}

Which leaves us with the following expression for the reflection coefficient
\begin{equation*}
\label{eq:ApproxReflecCoeff}
    \Gamma \approx \frac{2Y_{0}}{RS^2 + Y_{0}} - 1
\end{equation*}

And this one for the contrast
\begin{equation*}
\label{eq:ParallelContr}
    |\Delta\Gamma| = |\Gamma(R=\ROff) - \Gamma(R=\ROn)|
                   \approx 2Y_{0}\left|\frac{1}{
                   \ROff S^2 + Y_{0}} - \frac{1}{\ROn S^2 + Y_{0}
               }\right|
\end{equation*}

Now, thanks to this simplified form of the contrast, to obtain the optimum
value for \(C_{c}\) we don't need any fancy tricks, just to derive with respect
to \(C_{c}\) and equate to \(0\). Doing this we arrive at the equation
\begin{equation}
\label{eq:ParallelContrOptS2NonRho}
    S^2 = \frac{Y_{0}}{\sqrt{\ROff\ROn}}
\end{equation}

And solving it for \(C_{c}\), we get the single solution
(for \(\ROn, \ROff, Y_{0}, L, C_{p}, C_{c} \geq 0\))

\begin{equation}
\label{eq:ParallelContrOptCcNonRho}
C_{c\text{Max}} = \frac{L Y_{0}}{2\sqrt{\ROff\ROn}}
\left(1 + \sqrt{1 + 4C_{p}\frac{\sqrt{\ROff\ROn}}{LY_{0}}}\right)
\end{equation}

We could simply plug this result into a simulation and call it a day, but with
a little bit more digging we can extract some interesting results.

First off, by the way the resistances appear in \ref{eq:ParallelContrOptS2NonRho}
and \ref{eq:ParallelContrOptCcNonRho} it leads really naturally to defining
a ratio parameter

\begin{equation*}
\label{eq:RhoDef}
    \rho = \frac{\ROff}{\ROn} \geq 1
\end{equation*}

With it our expressions \ref{eq:ParallelContrOptS2NonRho} and
\ref{eq:ParallelContrOptCcNonRho} turn to

\begin{equation}
\label{eq:ParallelContrOptS2}
    S^2 = \frac{Y_{0}}{\sqrt{\rho}\ROn}
\end{equation}

\begin{equation*}
\label{eq:ParallelContrOptCc}
C_{c\text{Max}} = \frac{L Y_{0}}{2\sqrt{\rho}\ROn}
\left(1 + \sqrt{1 + 4C_{p}\frac{\sqrt{\rho}\ROn}{LY_{0}}}\right)
\end{equation*}

Then, by using the definition of \(S\) from \ref{eq:ImpParallelResonant} and
using impedance, we can rearrange \ref{eq:ParallelContrOptS2} to

\begin{equation*}
\label{eq:ParallelTunning}
Z_{0} = \frac{L(C_{c}+C_{p})}{\sqrt{\rho}\ROn C_{c}^2}
\end{equation*}

Which is what we saw in figure \ref{eq:ReflecCoeff} with the
variations in resistance: When tuning the resonator to the geometric mean of
the resistances of the two states, the reflection coefficients end up in
opposing sides of 0 within the real line. What this result shows is that
this is the optimum arrangement.

In addition to this insight, we can also use \ref{eq:ParallelContrOptS2} in
our approximation for the contrast (\ref{eq:ParallelContrAprox}) to see
that, when optimized, it only depends on the ratio of the resistances
(or the ration between the resistances and the
tuning resistance), and that it has a maximum value of 2, which is expected:
\begin{equation}
\label{eq:OptimumParallelContr}
    |\Delta\Gamma| \approx 2Y_{0}\left|\frac{1}{
                   \ROff S^2 + Y_{0}} - \frac{1}{\ROn S^2 + Y_{0}
               }\right|
                   = 2\left|
                   \frac{1}{\sqrt\rho + 1} - \frac{\sqrt\rho}{1 + \sqrt\rho}
                   \right|
                   = 2\left|
                   \frac{1 - \sqrt\rho}{1 + \sqrt\rho}
                   \right|
\end{equation}

After these results it seems appropriate to analyze with more detail the
approximations used, so we can contextualize the regime in which this works.
The first approximation done was \(\omega \approx \omega_{r}\), which boils
down to \ref{eq:ApproxLWr} and \ref{eq:ApproxCcWr}. The second approximation
was \ref{eq:ParallelContrAprox}, so let's see if with the optimum \(C_{c}\) it
holds. Using \ref{eq:ParallelContrOptS2} we have
\begin{equation*}
    (RS^2 + Y_{0})^2 \gg S^2 \rightarrow
    \left(\frac{R Y_{0}}{\sqrt\rho \ROn} + Y_{0}\right)^2 \gg
    \frac{Y_{0}}{\sqrt\rho \ROn}
\end{equation*}

Now, considering \(R = \ROn\) since it's the worst case scenario and
returning to the use of impedance instead of admittance, the condition turns to
\begin{equation*}
\label{eq:ProtoOptParallelContrApprox}
    \left(\frac{1}{\sqrt\rho} + 1\right)^2 \gg
    \frac{Z_{0}}{\sqrt\rho \ROn}
\end{equation*}

Finally, we can use the worst possible value of \(\rho\) on each side
(\(\rho = 1\) for the right side and \(\pi = \infty\) for the left) and
arrive at
\begin{equation}
\label{eq:OptParallelContrApprox}
    \ROn \gg Z_{0}
\end{equation}

Now that we have finished our calculations, it is time to introduce the
parasitic resistance. This is important because it depends on the design of
the circuit, and plotting \(|\Delta\Gamma|(R_{p})\) will tell us with want
amount of losses we can get our results.

To introduce it we will give it the same treatment to \(R_{p}\) as to
\(\ROff\) by introducing a ratio parameter
\begin{equation*}
\label{eq:PiDef}
    \pi = \frac{R_{p}}{\ROn^{SET}}
\end{equation*}

With the important distinction that, as opposed to \(\rho\), \(0\leq\pi\leq\infty\),
with \(\pi = 0\) being a short circuit and \(\pi = \infty\)
being a lossless circuit.

The substituions are then

\begin{align*}
    R_{\text{On}}
    &= \frac{\pi}{1 + \pi} \ROn^{SET}\\
    R_{\text{Off}}
    &= \frac{\rho_{SET}\pi}{\rho_{SET} + \pi}\ROn^{SET}\\
    \rho
    &= \frac{\rho_{SET}(1 + \pi)}{\rho_{SET} + \pi}
\end{align*}

With \(\rho_{SET} = \ROff^{SET}/\ROn^{SET}\). Taking this even further beyond
with the fact that in an SET \(\rho_{SET} \approx \infty\)
(in the off state, no electrons are travelling through), the substitutions are

\begin{align*}
    \ROn
    &= \frac{\pi}{1 + \pi} \ROn^{SET}\\
    \ROff
    &= \pi \ROn^{SET}\\
    \rho
    &= 1 + \pi
\end{align*}

The introduction of the parasitic resistance and an infinite \(\ROff^{SET}\)
doesn't change much, in the sense that for most of the expressions is better
to simply use \(\rho\) and \(\ROn\) for clarity. Most. Because for two results
in specific it helps: in \ref{eq:OptimumParallelContr}

\begin{equation}
\label{eq:OptimumParallelContrOfPi}
    |\Delta\Gamma| \approx
                   2\left|
                   \frac{1 - \sqrt{1 + \pi}}{1 + \sqrt{1 + \pi}}
                   \right|
\end{equation}

And in \ref{eq:OptParallelContrApprox}

\begin{equation}
\label{eq:OptParallelContrApproxOfPi}
\frac{\pi}{1 + \pi}\ROn^{SET} \gg Z_{0} \rightarrow
\pi \gg \frac{Z_{0}}{\ROn^{SET} - Z_{0}}
\end{equation}

Using the values of \(Z_{0}\) and \(\ROn^{SET}\) that we have been considering
up until now (\(50\unit{\ohm}\) and \(50\unit{\kilo\ohm}\) respectively) we can
see that for \ref{eq:OptParallelContrApprox} to work in a worse case scenario,
\(\pi\) must be a lot greater than \(10^{-3}\). It probably would
be, given that for a \(\pi\) 100 times greater,
\(|\Delta\Gamma| \approx 0.0477\), which isn't a good contrast to
aim for.

Finally, we check our results by comparing them against the numerically
calculated optimum contrast via a simulation that searches the optimum
value of \(C_{c}\) for a given value of \(\pi\).

\begin{figure}[t]
\centering
  \includesvg[width=\linewidth]{RLCParallel/OptContrComparison.svg}
  \caption{Numerical optimum contrast and our formula (\ref{eq:OptimumParallelContrOfPi})
  with the difference between them.
\(L = 180\unit{\nano\henry}\), \(C_{p} = 500\unit{\femto\farad}\),
\(Z_{0} = 50\unit{\ohm}\), \(\ROn^{SET} = 50\unit{\kilo\ohm}\),
\(\rho = 2\cdot10^{6}\). The script generating these images can be found
in the companion Github repository with the name \texttt{OptContrComparison.py}}
\label{fig:ParallelContrComparison}
\end{figure}

As we can see in figure \ref{fig:ParallelContrComparison}, our
estimations for what values of \(\pi\) were completely off, but in a good way.
That is because even though the approximation gets worse for greater \(\pi\),
it caps off at \(0.0175\), which is negligible in comparison to the values of
\(|\Delta\Gamma|\) that we are aiming to.

To summarize, we have found the expression for the value of \(C_{c}\) that
optimizes the contrast. This value effectively tunes the resonator to the
geometric mean of the 2 possible resistances of the circuit, and makes
the contrast depend on only in their ratio. All of this was found with
the constraints \ref{eq:ApproxLWr}, \ref{eq:ApproxCcWr} and
\ref{eq:ApproxReflecCoeff}.

\end{document}
