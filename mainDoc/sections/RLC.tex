\documentclass[../main.tex]{subfiles}

\begin{document}

Now that we have a good theoretical context of all the parts of the problem,
we will start by analyzing the resonator with a non-kinetic inductance.

\begin{figure}
\centering
\begin{circuitikz}[]
    \draw (-0.5,0) to[short]
          (0,0) to[generic, l=\(Z_0\)]
          (1.5,0) to[capacitor, l=\(C_c\)]
          (3,0) to[short]
          (6,0);
    \draw (3,0) to[L, l=\(L\)]
          (3,-2);
    \draw (4.5,0) to[capacitor, l=\(C_p\)]
          (4.5,-2);
    \draw (6,0) to[R, l=\({R = R_{SET} \parallel R_p}\)]
          (6,-2);
    \draw (3,-2) to[short]
          (6,-2) node[ground]{};
\end{circuitikz}
\caption{Topology of the resonator that we are going to use. \(C_{p}\) and
\(R_{p}\) are a virtual capacitor and resistance used to model losses in the
circuit, while \(R_{SET}\) is the resistance of the SET in any state. That
leaves \(C_{c}\) and \(L\) as the degrees of freedom in our system.}
\label{fig:RLC}
\end{figure}

\subsection{Resonant frequency and effective impedance}
Our analysis begins with obtaining expressions for the resonant frequency and
the effective impedance of our resonator. It's easy to see that the impedance
of our resonator in figure \ref{fig:RLC} is

\begin{equation}
\label{eq:ImpParallel}
    Z = \frac{1}{j \omega C_{p} + \frac{1}{j \omega L} + \frac{1}{R}}
        + \frac{1}{j \omega C_{c}}
\end{equation}

Which after a little massaging turns into

\begin{equation}
\label{eq:ImpParallelBinomial}
    Z = \frac{\omega^2 L^2 R}{R^2(1-\omega^2C_{p}L)^2 + \omega^2 L^2} +
        j \left(
            \frac{\omega^2 L^2 R}{R^2(1-\omega^2C_{p}L)^2 + \omega^2 L^2}
            - \frac{1}{\omega C_{c}}
          \right)
\end{equation}

The resonant frequency \(\omega_{r}\) that makes \(\Im Z = 0\) is

\begin{equation}
\label{eq:ExactWr}
    \omega_{r}^2 = \frac{1}{L(C_{c} + C_{p})}
    \left(
        1 + \frac{C_{c}}{2C_{p}} - \frac{L}{2 R^2 C_{p}} \pm
        \sqrt{\left(1 + \frac{C_{c}}{2C_{p}} - \frac{L}{2 R^2 C_{p}}\right)^2
        - 1 - \frac{C_{c}}{C_{p}}}
    \right)
\end{equation}

Choosing \(C_{c}\) and \(L\) such that
\(\frac{C_{c}}{C_{p}}, \frac{L}{R^2 C_{p}} \ll 1\), leaves us with the
approximate expression for the resonant frequency

\begin{equation}
\label{eq:Wr}
\omega_{r} \approx \frac{1}{\sqrt{L (C_{c} + C_{p})}}
\end{equation}

Finally, to obtain the effective impedance we use this expression in \(\Re Z\)

\begin{equation}
\label{eq:ExactZeff}
    Z_{eff} = \Re Z(\omega_{r}) =
    \frac{\omega_{r}^2 L^2 R}{R^2(1-\omega_{r}^2C_{p}L)^2 + \omega_{r}^2 L^2}
    \approx \frac{L (C_{c} + C_{p})}{R C_{c}^2}
      \left(1 + \frac{L (C_{c} + C_{p})}{R^2 C_{c}^2}\right)^{-1}
\end{equation}

And by, again, choosing \(L\) and \(C_{c}\) such that
\(\frac{L (C_{c} + C_{p})}{R^2 C_{c}^2} \ll 1\) we arrive to our expression for
the effective impedance

\begin{equation}
\label{eq:Zeff}
    Z_{eff} \approx \frac{L (C_{c} + C_{p})}{R C_{c}^2}
\end{equation}

In future sections we will be using quite a lot of expressions obtained via
approximations in non-approximated systems, only to do more approximations
with them. Due to this, it is really important to have a clear picture of
the regimes we are working in to ensure that our results work in the
state-of-the-art technology, and that's why after each result we are going to
recontextualize our approximations.

In this case, the approximations to obtain \(\omega_{r}\) are clear and
straight forward:

\begin{gather}
    \frac{C_{c}}{C_{p}} \ll 1 \label{eq:ApproxCcWr}\\
    \frac{L}{R^2 C_{p}} \ll 1 \label{eq:ApproxLWr}
\end{gather}

But the approximation for \(Z_{eff}\) needs a little bit of extra work. If we
multiply \((C_{c} / C_{p})^2\) in both sides, it turns into

\begin{equation}
\label{eq:ProtoZeffCond}
    \frac{L}{R^2 C_{p}} \left(1 + \frac{C_{c}}{C_{p}}\right) \ll
    \left(\frac{C_{c}}{C_{p}}\right)^2
\end{equation}

And since we used equation \ref{eq:Wr} to arrive here, it must hold
the approximation \ref{eq:ApproxCcWr}, turning the previous expression into

\begin{equation}
\label{eq:ApproximationZeff}
    \frac{L}{R^2 C_{p}} \ll \left(\frac{C_{c}}{C_{p}}\right)^2
\end{equation}

While approximations \ref{eq:ApproxCcWr} and \ref{eq:ApproxLWr} impose a
\textbf{general condition} in our degrees of freedom, approximation
\ref{eq:ApproximationZeff} imposes a \textbf{relative condition} between
the previous two.

\subsection{Contrast and it's optimization}
\lipsum[1-1]
\end{document}
