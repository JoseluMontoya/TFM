\documentclass[../main.tex]{subfiles}

\begin{document}
Some obvious next steps would be the experimental measurement of
the optimum parameters we found and of the effect of the kinetic inductance.

Another good direction would be to try and obtain analytical expressions for
the contrast with a kinetic inductance. This would allow us to gain a deeper
knowledge, in the same vein as the optimization of the non-kinetic circuit.

Finally, an interesting idea to explore and formalize (credit to my supervisor for it,
Fernando Gonzalez Zalba) is to design the kinetic inductance in such a way that,
for the On state, the super conductance would brake, turning the inductor into a
resistance (ideally high enough to not create a short circuit).
Modifying the script used to generate figure \ref{fig:IdealContrast}
to plot this case, all tunings converge to the \(\LOff\), but with an
optimum \(C_{c}\) arbitrarily large for all cases. This case needs to be
studied with more care and more formally, since it appears to be the definitive
solution.

To end this thesis, I would like to thank my supervisor Fernando Gonzalez Zalba
for all the guidance and help he gave me, and to my family and significant
other for all the love and support.
\end{document}
