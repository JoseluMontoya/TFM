\documentclass[../main.tex]{subfiles}

\begin{document}
Silicon-based quantum computing is a quantum computing paradigm that uses
trapped electrons in silicon quantum dots to build the qubits. The 3 main
qubit architectures using QD are\cite{burkardSemiconductorSpinQubits2023}:

\begin{itemize}
    \item \textbf{Loss-DiVincenzo qubits:} Thought up by Loss and DiVincenzo\cite{lossQuantumComputationQuantum1998}
        in their original paper, it defines the qubits as the spins of the
        trapped electrons themselves, and for their control uses a combination
        of an external magnetic field and the control of tunnel barriers
        between the QD.
    \item \textbf{Singlet-triplet qubits:} Proposed by Levy\cite{levyUniversalQuantumComputation2002}, instead
        of using single spins, it uses as basis the singlet and triplet state
        of a pair of spins. The main benefit with respect to the Loss-DiVincenzo
        is that the qubit lives in a decoherence-free subspace with respect
        to global magnetic fields that couple to the spin of the electron.
    \item \textbf{Exchange-only qubits:} Created by DiVincenzo \textit{et al.}\cite{divincenzoUniversalQuantumComputation2000},
        it uses
        the spin of 3 QD to create a qubit fully controllable using voltages
        through the tunnel barriers that separate them.
\end{itemize}

The main advantage of silicon-based quantum computing is scalability, since
it can leverage the decades in advancements of the CMOS industry. The objective
with silicon-based quantum computing is then to build a quantum computer with
an enormous amount of high fidelity qubits, instead of a quantum computer
with a great amount of extremely high fidelity qubits, and to use the bigger
number of qubits to run error correction algorithms. For this to be implemented,
we need high fidelity and high speed readout techniques. At the moment there are
multiple measuring schemes showing promising results\cite{gonzalez-zalbaScalingSiliconbasedQuantum2021}, such as RF reflectometry
of a SET. In this master's thesis, we will explore the use of a kinetic inductor
in this measuring scheme with the objective of improving the performance.
\end{document}
