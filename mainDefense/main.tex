\documentclass{beamer}

\usetheme{Antibes}
% \usefonttheme[onlymath]{serif}
\usefonttheme{serif}
% %Basics
\usepackage{inputenc}
\usepackage[T1]{fontenc}
\usepackage{float}
% \usepackage{enumitem}
% \usepackage[spanish, es-noquoting]{babel}
\usepackage[]{babel}

% % Visual flare and structure
\usepackage{tikz}
\usepackage{circuitikz}
\usepackage{graphicx}
\usepackage[]{geometry}
% \usepackage[pagestyles, explicit]{titlesec}
\usepackage{caption}
\usepackage{subcaption}
\usepackage{soul}
% \usepackage{multicol}
% \usepackage{abstract}
\usepackage{appendix}
\usepackage{lipsum}
\usepackage{svg}

% % Math
\usepackage{bm}
\usepackage{amsmath}
\numberwithin{equation}{section}
\usepackage[makeroom]{cancel}
\usepackage{empheq}
\usepackage{physics}
\usepackage{siunitx}

\AtBeginDocument{\RenewCommandCopy\qty\SI}
\ExplSyntaxOn
\msg_redirect_name:nnn { siunitx } { physics-pkg } { none }
\ExplSyntaxOff

% % Links
\usepackage[]{hyperref}
\definecolor{cerulean}{rgb}{0.0, 0.48, 0.65}
\hypersetup{
    colorlinks=true,
    linkcolor=cerulean,
    filecolor=magenta,
    urlcolor=cyan,
    citecolor=magenta,
    % pdftitle={Overleaf Example},
    % pdfpagemode=FullScreen,
    }

% % Subfiles
\usepackage{subfiles}

% % Tikz libraries
\usetikzlibrary{babel}
\usetikzlibrary{calc}
\usetikzlibrary{math}
\usetikzlibrary{backgrounds}

\renewcommand{\CancelColor}{\color{red}}

\parskip=1em

\newcommand{\ROn}{R_{\text{On}}}
\newcommand{\ROff}{R_{\text{Off}}}
\newcommand{\LOn}{L_{\text{On}}}
\newcommand{\LOff}{L_{\text{Off}}}

\graphicspath{{./../code/img/}}

\title{
\textbf{SET readout using RF reflectometry and kinetic inductance nonlinearity}
}
\author{Jose Luis Montoya Agull\'o}
\date{}

\AtBeginSection[]
{
\begin{frame}
    % \frametitle{Table of Contents}
    \tableofcontents[currentsection,currentsubsection,sectionstyle=show/shaded,subsectionstyle=hide/hide/hide,subsubsectionstyle=hide/hide/hide]
\end{frame}
}

\AtBeginSubsection[]
{
\begin{frame}
    \tableofcontents[currentsection,currentsubsection,sectionstyle=show/hide,subsectionstyle=show/shaded/hide,subsubsectionstyle=hide/hide/hide]
\end{frame}
}

\AtBeginSubsubsection[]
{
\begin{frame}
    \tableofcontents[currentsection,currentsubsection,sectionstyle=hide/hide,subsectionstyle=show/hide,subsubsectionstyle=show/shaded/hide]
\end{frame}
}


\begin{document}
\begin{frame}
    \maketitle
\end{frame}

% \newgeometry{top=2cm}
% \maketitle
% \begin{abstract}
%     In this thesis we will make a proof of concept of a new type of charge sensor for quantum computation,
%     using an SET, RF reflectometry and a kinetic inductance, which is the novel
%     part of the sensor. First we will optimize the sensor analytically using a non-kinetic inductance,
%     and then we will simulate the kinetic version to see if we can improve the fidelity.
%     The results show that, for a non-kinetic sensor, the optimum tuning is done to
%     the geometric mean of the two resistances we want to distinguish between, and
%     that the use of a kinetic inductor with a proper calibration increases greatly
%     the fidelity of the sensor, specially with parasitic resistances of the order
%     of the smallest resistance.
% \end{abstract}
%
%

\begin{frame}{Meaning of the title}
\begin{itemize}
    \item \textbf{SET:} Single electron transistor
    \item \textbf{RF reflectometry:} Measuring method for the SET
    \item \textbf{Kinetic inductance:} Inductance using kinetic energy
\end{itemize}
\end{frame}

\begin{frame}{Outline of defense}
\setcounter{tocdepth}{1}
\tableofcontents{}
\setcounter{tocdepth}{3}
\end{frame}

% \restoregeometry{}
%
% \newpage
\section{Introduction}
\label{sec:Intro}
\subfile{sections/Introduction.tex}

\section{Aims and objectives}
\label{sec:Objectives}
\subfile{sections/Objectives.tex}

\newpage
\section{Theoretical background}
\label{sec:Theory}
\subfile{sections/SET.tex}
\subfile{sections/RfReflectometry.tex}
\subfile{sections/KineticInductance.tex}

\newpage
\section{Results}
\label{sec:Results}
\begin{frame}{Measuring circuit}
    \begin{figure}
    \centering
    \begin{circuitikz}[scale=0.9]
        \draw (-0.5,0) to[short]
              (0,0) to[generic, l=\(Z_0\)]
              (1.5,0) to[capacitor, l=\(C_c\)]
              (3,0) to[short]
              (6,0);
        \draw (3,0) to[L, l=\(L\)]
              (3,-2);
        \draw (4.5,0) to[capacitor, l=\(C_p\)]
              (4.5,-2);
        \draw (6,0) to[vR, l=\({R = R_{SET} \parallel R_p}\)]
              (6,-2);
        \draw (3,-2) to[short]
              (6,-2) node[ground]{};
    \end{circuitikz}
    % \caption{Topology of our measuring circuit. Since during most of our
    % analysis we will ignore \(R_{p}\), it is already couplet to \(R_{SET}\).}
    \label{fig:RLC}
    \end{figure}
\end{frame}


\subsection{The parallel RLC resonator}
\label{subsec:ParallelRLC}
\subfile{sections/RLC.tex}

\subsection{The parallel kinetic RLC resonator}
\label{subsec:KineticRLC}
\subfile{sections/KineticRLC.tex}

\section{Conclusions}
\label{sec:Conclusions}
\subfile{sections/Conclusions.tex}

\section{Next steps and closing remarks} %Vistas de futuro: que hacer despues
\label{sec:NextSteps}
\subfile{sections/NextSteps.tex}

% \nocite{zhaoPhysicsSuperconductingTravellingWave2022}
% \bibliographystyle{plain}
% \bibliography{../TFM}

% \newpage
% \appendix \appendixpage \addappheadtotoc
%
% \section{Cuentas}
% \subfile{sections/Ap_1}

\end{document}

