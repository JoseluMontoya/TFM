\documentclass[../main.tex]{subfiles}

\begin{document}
% We have been able to optimize the matching network attached to an SET. It turns
% out that for an optimum measurement, we need to tune the network for the
% geometric mean of the two resistances of the SET. These finding could prove
% extremely useful since, to my knowledge, analytical expressions for the
% optimum contrast and for the parameters that enabled that contrast were not
% derived up until now.
%
% In addition to this, we showed that the use of a kinetic inductor improves
% greatly the quality of measurements, with the only recommendations for the
% implementation being:
% \begin{itemize}
%     \item The larger the variation in inductance, the better
%     \item The best tuning to use will depend on your restrictions:
%         \(\LOn\) tuning will be best if you need the lowest possible frequency,
%         and you have some losses. If your concern is that you need a greater
%         value of \(C_{c}\), the middle tuning is perfect because, even though
%         it uses a higher frequency than the \(\LOn\) tuning, it needs the
%         greatest \(C_{c}\) and the performance is better than the non-kinetic
%         case for all values of \(\pi\). Finally, if you want the best of the
%         best (mostly when the losses practically non-existent, and
%         you need a sliver of improvement), and you can afford a higher frequency
%         and a small \(C_{c}\), \(\LOff\) tuning is the best.
% \end{itemize}
% Due to the biggest improvements being around \(\pi = 1\), I believe that
% the value of these finding lies in that the improvements of a kinetic inductance
% allow for bigger losses, and limit the circuit design a lot less.
\begin{frame}{Conclusions}
\begin{itemize}
    \item \textbf{Analytical} optimization of the non-kinetic measuring circuit
        \begin{itemize}
            \item Tuning to geometric mean of resistances
            \item Optimum contrast only depends on \(\rho\)/\(\pi\)
        \end{itemize}
    \item Showed kinetic inductance objectively improves SNR with appropriate
        configuration
        \begin{itemize}
            \item \(\LOn\) tuning: Low \(w\) + some losses
            \item Middle tuning: Greatest \(C_{c}\) + virtually max performance
            \item \(\LOff\) tuning: Max performance at greater \(\pi\)
        \end{itemize}
    \item Most improvement at around \(\pi = 1\), so best for giving freedom in
        circuit design
\end{itemize}
\end{frame}

\end{document}
