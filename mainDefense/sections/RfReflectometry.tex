\documentclass[../main.tex]{subfiles}

\begin{document}
\subsection{RF reflectometry}
\label{subsec:RFRef}
% Radio frequency reflectometry is a method to measure change in an impedance
% connected to a transmission line via the reflection of a radio signal.
%
% Usual lumped element treatment of AC circuits assumes that the size of the
% circuit is small with respect to the wavelength of the voltage,
% but with radio frequency voltages we cannot do that. The main consequences for
% us are:

% \begin{itemize}
%     \item Now voltage and current are dependent on how far along the circuit
%     you are, due to how fast it changes with respect to the size of the circuit
%     itself (\(V(t), I(t) \rightarrow V(x, t), I(x, t)\))
% \item The intrinsic inductance and capacitance per unit length of a long
%     connection cannot be ignored
% \end{itemize}

% \begin{frame}{Things to keep in mind with AC RF}
% \begin{itemize}
%     \item \(V(t), I(t) \rightarrow V(x, t), I(x, t)\)
% \item Intrinsic inductance/capacitance cannot be ignored
% \end{itemize}
% \end{frame}

% In such cases those connections are made with what is known as a
% transmission line, which is a cable designed to minimize the radiation of
% power via that inductance and capacitance, and includes a signal and a
% ground connection in one package. One example of a transmission line would be a
% coaxial cable, in which the central conductor is the signal and the outer jacket
% ground.

% \begin{frame}{Transmission line circuit}
% \begin{figure}[t]
% \centering
% \begin{circuitikz}[]
%     \newcommand{\lenghOfTransmissionLine}{4}
%     \draw (0,0) node[ground]{};
%     \draw (0,0) to[sV] ++(0,1.5) to[short] ++(1.5,0);
%     \draw (1.5,0) to[capacitor] ++(0,1.5) to[L] ++(1.5,0) to[short] ++(0.5,0);
%     \foreach \i in {2,...,\lenghOfTransmissionLine}{
%         \draw (2*\i - 0.5,0) to[capacitor] ++(0,1.5) to[L] ++(1.5,0) to[short] ++(0.5,0);
%     }
%     \draw ({2*(1+\lenghOfTransmissionLine)-0.5},0) node[ground]{};
%     \draw ({2*(1+\lenghOfTransmissionLine)-0.5},1.5) to[generic, l=\(Z\)] ++(0,-1.5);
%     \draw (0,0) to[short] ++({2*(1+\lenghOfTransmissionLine)-0.5},0);
%     \draw[dashed] (0.93,-0.15) rectangle ++(2,2);
%     \draw [latexslim-latexslim] (2.93,-0.35) -- node[below] {\(\Delta x\)} ++(-2,0);
% \end{circuitikz}
% % \caption{Lumped element model of a lossless transmission line connected to a
% %     generic impedance \(Z\). In order to model the relevant
% %     impedance per unit length of the transmission line at radio frequencies,
% %     we represent it via sections with inductors in series and capacitors
% %     in parallel. Each periodic section like the one inside the dashed rectangle
% %     represents a segment of length \(\Delta x\) of the transmission line.}
% \label{fig:LumpedTransmisionLine}
% \end{figure}
% \end{frame}

% Even though a transmission line minimizes that radiation of power, it does not
% erase it, and we need to take it into account in our calculations. To model this
% inductance and capacitance per unit length (\(L_{l}\) and \(C_{l}\) respectively),
% we will discretize it via a lumped element representation like in figure
% \ref{fig:LumpedTransmisionLine}, ignoring the losses by not including any
% resistance  in our circuit.
% Each pair inductor-capacitor will occur along a length \(\Delta x\) of the
% transmission line, so their inductance and capacitance in that stretch will be
% \(L = \Delta x \cdot L_{l}\) and \(C = \Delta x \cdot C_{l}\). Applying Kirchhoff
% at a point \(x\) in the transmission line we get

% \begin{align*}
%     V(x+\Delta x, t) - V(x, t) &= -\Delta x L_{l} \frac{\partial I}{\partial t}\\
%     I(x+\Delta x, t) - I(x, t) &= -\Delta x C_{l} \frac{\partial V}{\partial t}\\
% \end{align*}


% And doing the limit \(\Delta x \to 0\) gives us the telegraph equations for a
% lossless transmission line
% \begin{align}
% \begin{split}
% \label{eq:TelegraphEq}
%     \frac{\partial V}{\partial x} = - L_{l}\frac{\partial I}{\partial t}\\
%     \frac{\partial I}{\partial x} = - C_{l}\frac{\partial V}{\partial t}\\
% \end{split}
% \end{align}

% \begin{frame}{Telegraph equations}
% \begin{gather*}
%     \begin{split}
%         V(x+\Delta x, t) - V(x, t) &= -\Delta x L_{l} \frac{\partial I}{\partial t}\\
%         I(x+\Delta x, t) - I(x, t) &= -\Delta x C_{l} \frac{\partial V}{\partial t}\\
%     \end{split}\\
%     \Downarrow\\
%     \frac{\partial V}{\partial x} = - L_{l}\frac{\partial I}{\partial t}\\
%     \frac{\partial I}{\partial x} = - C_{l}\frac{\partial V}{\partial t}\\
% \end{gather*}
% \end{frame}

% With solutions
% \begin{align*}
% \begin{split}
%     \label{eq:TelegraphSol}
%     V(x,t) &= V_{+}(t - x/v_{p}) + V_{-}(t + x/v_{p})))\\
%     I(x,t) &= \frac{1}{Z_{0}}(V_{+}(t - x/v_{p}) - V_{-}(t + x/v_{p}))
% \end{split}
% \end{align*}

% \begin{frame}{Telegraph solutions}
% \begin{align*}
% \begin{split}
%     \label{eq:TelegraphSol}
%     V(x,t) &= V_{+}(t - x/v_{p}) + V_{-}(t + x/v_{p})))\\
%     I(x,t) &= \frac{1}{Z_{0}}(V_{+}(t - x/v_{p}) - V_{-}(t + x/v_{p}))
% \end{split}
% \end{align*}
% \end{frame}

% Where \(v_{p} = \frac{1}{\sqrt{L_{l}C_{l}}}\) is the phase velocity of the wave,
% \(Z_{0} = \sqrt{\frac{L_{l}}{C_{l}}}\) is the characteristic impedance of the
% line and \(V_{+}\) and \(V_{-}\) are generic functions that describe a right
% and left traveling wave respectively. Since we will be choosing our reference
% frame such that our signal will be always traveling from left to right, the
% appearance of \(V_{-}\) in our calculations will mean a reflection.
%
% If we add a generic impedance \(Z\) at the end of the transmission line, we
% add the boundary condition
% \begin{equation*}
% \label{eq:RfBoundCond}
%     \frac{V(x_{\text{End}},\omega)}{I(x_{\text{End}},\omega)} = Z(\omega)
% \end{equation*}
% With \(V(x,\omega)\) and \(I(x,\omega)\) being the time Fourier transforms of
% \(V(x,t)\) and \(I(x,t)\) respectively. Choosing \(x=x_{\text{End}}=0\) to
% simplify and using the time Fourier transforms of expressions
% \ref{eq:TelegraphEq}, we get
% \begin{equation}
% \label{eq:TelegraphEqFourierWithBound}
%     \frac{V(0,\omega)}{I(0,\omega)} =
%     Z_{0}\frac{V_{+}(\omega) + V_{-}(\omega)}
%     {V_{+}(\omega) - V_{-}(\omega)} = Z(\omega)
% \end{equation}

% \begin{frame}{Boundary condition}
% \begin{gather*}
%     \frac{V(x_{\text{End}},\omega)}{I(x_{\text{End}},\omega)} = Z(\omega)\\
%     \Downarrow\\
%     \frac{V(0,\omega)}{I(0,\omega)} =
%     Z_{0}\frac{V_{+}(\omega) + V_{-}(\omega)}
%     {V_{+}(\omega) - V_{-}(\omega)} = Z(\omega)
% \end{gather*}
% \end{frame}

%
% As we can see, the only way in which \(Z_{0} = Z(\omega)\) is if
% \(V_{+,-}(\omega)=0\), or in other words, the only way to not get a reflection
% is to match \(Z_{0}\) to \(Z(\omega)\). A useful parameter to define is the
% reflection coefficient \(\Gamma = \frac{V_{-}(\omega)}{V_{+}(\omega)}\), and
% with equality \ref{eq:TelegraphEqFourierWithBound} has the form
% \begin{equation}
% \label{eq:ReflecCoeff}
%     \Gamma = \frac{Z(\omega) - Z_{0}}{Z(\omega) + Z_{0}}
% \end{equation}
%
% This reflection coefficient is the key to RF reflectometry, because if we
% know the characteristic impedance of our transmission line, we can measure
% the power and the phase reflected and obtain \(Z(\omega)\).
%
% If we want to measure 2 distinct impedances, and we want to maximize the
% reliability of our measure, the obvious way to do it would be to make the values
% of \(\Gamma\) be as separated as possible. A more formal way to express this
% idea is through the signal-to-noise ratio (SNR)\cite{vigneauProbingQuantumDevices2023} of the measurement
%
% \begin{equation}
% \label{eq:SNR}
% \text{SNR} = |\Delta\Gamma|^2\frac{P_{0}}{P_{N}}
% \end{equation}

\begin{frame}{Reflection coefficient and SNR}
\begin{align*}
    \Gamma = \frac{Z(\omega) - Z_{0}}{Z(\omega) + Z_{0}} && 
    \text{SNR} = |\Delta\Gamma|^2\frac{P_{0}}{P_{N}}
\end{align*}
\end{frame}

% With \(P_{0}\) and \(P_{N}\) the power of the signal and noise respectively and
% \(\Delta\Gamma = \Gamma_{B} - \Gamma_{A}\) the difference in reflection
% coefficients between the two states to measure, whose modulus is what is
% called the contrast. If we want to improve our measurements, we need to
% increase our SNR.
\end{document}
