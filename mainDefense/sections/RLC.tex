\documentclass[../main.tex]{subfiles}

\begin{document}
% \subsubsection{Resonant frequency and effective impedance}

\begin{frame}{Effective frequency and impedance}
\begin{gather*}
    Z(\omega) = \frac{1}{j \omega C_{p} + \frac{1}{j \omega L} + \frac{1}{R}}
        + \frac{1}{j \omega C_{c}}\\
    \Downarrow\\
    \omega_{r} = \frac{1}{\sqrt{L(C_{c} + C_{p})}} \text{ with } \frac{C_{c}}{C_{p}}, \frac{L}{R^2 C_{p}} \ll 1 \\
    Z_{eff} = \frac{L(C_{c}+C_{p})}{R C_{c}^2} \text{ with } \frac{L}{R^2 C_{p}} \ll \left(\frac{C_{c}}{C_{p}}\right)^2
\end{gather*}
\end{frame}

% \begin{frame}{Checking with a simulation}
% \begin{table}[H]
%     \centering
%     \begin{tabular}{c|c|c|c|c}
%         \(Z_{0}\) & \(R\) & \(C_{p}\) & \(C_{c}\) & \(L\) \\\hline
%         \(50\unit{\ohm}\) & \(50\unit{\kilo\ohm}\) & \(500\unit{\femto\farad}\) & \(100\unit{\femto\farad}\) & \(41.67\unit{\nano\henry}\)
%     \end{tabular}
% \end{table}
% \[
%     \omega_{r}  = 1.00654\unit{\giga\hertz}
% \]
% Also with \(R = 100\unit{\kilo\ohm}\), \(R = 25\unit{\kilo\ohm}\) and \(L = 41.25\unit{\nano\henry}\)
% \end{frame}

% \begin{frame}{Checking with a simulation}
% \begin{figure}[t]
% \centering
% \includesvg[width=0.84\linewidth]{RLCParallel/ReflecCoeff.svg}
% \end{figure}
% \end{frame}

% \begin{frame}{Checking with a simulation}
% \begin{figure}[t]
% \centering
% \includesvg[width=0.87\linewidth]{RLCParallel/ReflecPhase.svg}
% \end{figure}
% \end{frame}


\subsubsection{Contrast and optimization}

\begin{frame}{What variable to optimize for?}
\begin{align*}
    \omega = \frac{1}{\sqrt{L(C_{c} + C_{p})}}
\end{align*}
% \begin{align*}
%     L \gg C_{c}
% \end{align*}
\begin{center}
    \(L\) to select a suitable frequency, \(C_{c}\) as the optimization
    variable
\end{center}
\end{frame}

\begin{frame}{Optimum \(C_{c}\)}
\begin{align*}
    \omega = \frac{1}{\sqrt{L(C_{c} + C_{p})}}
\end{align*}
\begin{align*}
    Z(\omega) &= \frac{1}{j \omega C_{p} + \frac{1}{j \omega L} + \frac{1}{R}} + \frac{1}{j \omega C_{c}}\\
              &= \frac{1}{RS^2 + jS}
       \text{ with } S = \frac{C_{c}}{\sqrt{L(C_{c} + C_{p})}} = \omega C_{c}
\end{align*}
\end{frame}

%
% Using this expression to obtain the reflection coefficient,
% but using the admittance of the transmission line instead of the impedance
% (\(Y_{0} = 1/Z_{0}\)) yields
% \begin{align*}
% \begin{split}
%     \Gamma &= \frac{Z(\omega) - Z_{0}}{Z(\omega) + Z_{0}}
%             = \frac{Y_{0} - 1/Z(\omega)}{Y_{0} + 1/Z(\omega)}\\
%            &= \frac{2Y_{0}}{Y_{0} + 1/Z(\omega)} - 1 = \frac{2 Y_{0}}{RS^2 + Y_{0} + jS} - 1\\
%            &= 2 Y_{0}\frac{RS^2 + Y_{0} - jS}{(RS^2 + Y_{0})^2 + S^2} - 1
% \end{split}
% \end{align*}
%
% Since \(\omega \approx \omega_{r}\) then
% \(\Im Z(\omega) \approx 0\) and by extension \(\Im \Gamma \approx 0\), so
% \begin{equation*}
%     \Gamma \approx 2 Y_{0}\frac{RS^2 + Y_{0}}{(RS^2 + Y_{0})^2 + S^2} - 1
% \end{equation*}
%
% Next, using the parameters utilized for figure \ref{fig:ReflexCoeffAndPhase}
% to get a sense of the scale, it is safe to assume that the following
% approximation is correct
% \begin{equation}
% \label{eq:ParallelContrAprox}
%     (RS^2 + Y_{0})^2 \gg S^2
% \end{equation}
%
% Which leaves us with the following expression for the reflection coefficient
% \begin{equation*}
% \label{eq:ApproxReflecCoeff}
%     \Gamma \approx \frac{2Y_{0}}{RS^2 + Y_{0}} - 1
% \end{equation*}

\begin{frame}{Optimum \(C_{c}\)}
\begin{align*}
\begin{split}
    \Gamma &= \frac{Z(\omega) - Z_{0}}{Z(\omega) + Z_{0}}\\
           %  = \frac{Y_{0} - 1/Z(\omega)}{Y_{0} + 1/Z(\omega)}\\
           % &= \frac{2Y_{0}}{Y_{0} + 1/Z(\omega)} - 1 = \frac{2 Y_{0}}{RS^2 + Y_{0} + jS} - 1\\
           % &= 2 Y_{0}\frac{RS^2 + Y_{0} - jS}{(RS^2 + Y_{0})^2 + S^2} - 1\\
           % &\approx 2 Y_{0}\frac{RS^2 + Y_{0}}{(RS^2 + Y_{0})^2 + S^2} - 1
           % \text{ because } \omega \approx \omega_{r}\\
           % &\approx \frac{2 Y_{0}}{RS^2 + Y_{0}} - 1
           % \text{ because } (RS^2 + Y_{0})^2 \gg S^2
           &\approx \frac{2 Y_{0}}{RS^2 + Y_{0}} - 1
           \text{ with } \omega \approx \omega_{r}
           \text{ and } (RS^2 + Y_{0})^2 \gg S^2\\
           \\
    |\Delta\Gamma| &= |\Gamma(R=\ROff) - \Gamma(R=\ROn)|\\
                   &\approx 2Y_{0}\left|\frac{1}{
                   \ROff S^2 + Y_{0}} - \frac{1}{\ROn S^2 + Y_{0}
               }\right|
\end{split}
\end{align*}
\end{frame}

%
% And this one for the contrast
% \begin{equation*}
% \label{eq:ParallelContr}
%     |\Delta\Gamma| = |\Gamma(R=\ROff) - \Gamma(R=\ROn)|
%                    \approx 2Y_{0}\left|\frac{1}{
%                    \ROff S^2 + Y_{0}} - \frac{1}{\ROn S^2 + Y_{0}
%                }\right|
% \end{equation*}

% \begin{frame}{Optimized contrast}
% \begin{align*}
%     |\Delta\Gamma| &= |\Gamma(R=\ROff) - \Gamma(R=\ROn)|\\
%                    &\approx 2Y_{0}\left|\frac{1}{
%                    \ROff S^2 + Y_{0}} - \frac{1}{\ROn S^2 + Y_{0}
%                }\right|
% \end{align*}
% \end{frame}

%
% Now, thanks to this simplified form of the contrast, to obtain the optimum
% value for \(C_{c}\) we don't need any fancy tricks, just to derive with respect
% to \(C_{c}\) and equate to \(0\). Doing this we arrive at the equation
% \begin{equation}
% \label{eq:ParallelContrOptS2NonRho}
%     S^2 = \frac{Y_{0}}{\sqrt{\ROff\ROn}}
% \end{equation}
%
% And solving it for \(C_{c}\), we get the single solution
% (for \(\ROn, \ROff, Y_{0}, L, C_{p}, C_{c} \geq 0\))
%
% \begin{equation}
% \label{eq:ParallelContrOptCcNonRho}
% C_{c\text{Max}} = \frac{L Y_{0}}{2\sqrt{\ROff\ROn}}
% \left(1 + \sqrt{1 + 4C_{p}\frac{\sqrt{\ROff\ROn}}{LY_{0}}}\right)
% \end{equation}

\begin{frame}{Optimum \(C_{c}\)}
\begin{gather*}
    \partial_{C_{c}} |\Delta\Gamma| = 0\\
    \Downarrow\\
    S^2 = \frac{Y_{0}}{\sqrt{\ROff\ROn}}\\
    \Downarrow\\
    C_{c\text{Max}} = \frac{L Y_{0}}{2\sqrt{\ROff\ROn}}
    \left(1 + \sqrt{1 + 4C_{p}\frac{\sqrt{\ROff\ROn}}{LY_{0}}}\right)
\end{gather*}
\end{frame}

%
% We could simply plug this result into a simulation and call it a day, but with
% a little bit more digging we can extract some interesting results.
%
% First off, by the way the resistances appear in \ref{eq:ParallelContrOptS2NonRho}
% and \ref{eq:ParallelContrOptCcNonRho} it leads really naturally to defining
% a ratio parameter
%
% \begin{equation*}
% \label{eq:RhoDef}
%     \rho = \frac{\ROff}{\ROn} \geq 1
% \end{equation*}
%
% With it our expressions \ref{eq:ParallelContrOptS2NonRho} and
% \ref{eq:ParallelContrOptCcNonRho} turn to
%
% \begin{equation}
% \label{eq:ParallelContrOptS2}
%     S^2 = \frac{Y_{0}}{\sqrt{\rho}\ROn}
% \end{equation}
%
% \begin{equation*}
% \label{eq:ParallelContrOptCc}
% C_{c\text{Max}} = \frac{L Y_{0}}{2\sqrt{\rho}\ROn}
% \left(1 + \sqrt{1 + 4C_{p}\frac{\sqrt{\rho}\ROn}{LY_{0}}}\right)
% \end{equation*}

\begin{frame}{Introduction of \(\rho\)}
\begin{gather*}
    \rho = \frac{\ROff}{\ROn} \geq 1\\
    S^2 = \frac{Y_{0}}{\sqrt{\rho}\ROn}\\
    C_{c\text{Max}} = \frac{L Y_{0}}{2\sqrt{\rho}\ROn}
    \left(1 + \sqrt{1 + 4C_{p}\frac{\sqrt{\rho}\ROn}{LY_{0}}}\right)
\end{gather*}
\end{frame}

%
% Then, by using the definition of \(S\) from \ref{eq:ImpParallelResonant} and
% using impedance, we can rearrange \ref{eq:ParallelContrOptS2} to
%
% \begin{equation*}
% \label{eq:ParallelTunning}
% Z_{0} = \frac{L(C_{c}+C_{p})}{\sqrt{\rho}\ROn C_{c}^2}
% \end{equation*}
%
% Which is what we saw in figure \ref{eq:ReflecCoeff} with the
% variations in resistance: When tuning the resonator to the geometric mean of
% the resistances of the two states, the reflection coefficients end up in
% opposing sides of 0 within the real line. What this result shows is that
% this is the optimum arrangement.
%
% In addition to this insight, we can also use \ref{eq:ParallelContrOptS2} in
% our approximation for the contrast (\ref{eq:ParallelContrAprox}) to see
% that, when optimized, it only depends on the ratio of the resistances
% (or the ration between the resistances and the
% tuning resistance), and that it has a maximum value of 2, which is expected:
% \begin{equation}
% \label{eq:OptimumParallelContr}
%     |\Delta\Gamma| \approx 2Y_{0}\left|\frac{1}{
%                    \ROff S^2 + Y_{0}} - \frac{1}{\ROn S^2 + Y_{0}
%                }\right|
%                    = 2\left|
%                    \frac{1}{\sqrt\rho + 1} - \frac{\sqrt\rho}{1 + \sqrt\rho}
%                    \right|
%                    = 2\left|
%                    \frac{1 - \sqrt\rho}{1 + \sqrt\rho}
%                    \right|
% \end{equation}

\begin{frame}{Optimized contrast}
\begin{gather*}
    Z_{0} = \frac{L(C_{c}+C_{p})}{\sqrt{\rho}\ROn C_{c}^2}\\
\begin{split}
    |\Delta\Gamma| &\approx 2Y_{0}\left|\frac{1}{
                   \ROff S^2 + Y_{0}} - \frac{1}{\ROn S^2 + Y_{0}
               }\right|\\
                   &= 2\left|
                   \frac{1}{\sqrt\rho + 1} - \frac{\sqrt\rho}{1 + \sqrt\rho}
                   \right|
                   = 2\left|
                   \frac{1 - \sqrt\rho}{1 + \sqrt\rho}
                   \right|
\end{split}
\end{gather*}
\end{frame}

%
% After these results it seems appropriate to analyze with more detail the
% approximations used, so we can contextualize the regime in which this works.
% The first approximation done was \(\omega \approx \omega_{r}\), which boils
% down to \ref{eq:ApproxLWr} and \ref{eq:ApproxCcWr}. The second approximation
% was \ref{eq:ParallelContrAprox}, so let's see if with the optimum \(C_{c}\) it
% holds. Using \ref{eq:ParallelContrOptS2} we have
% \begin{equation*}
%     (RS^2 + Y_{0})^2 \gg S^2 \rightarrow
%     \left(\frac{R Y_{0}}{\sqrt\rho \ROn} + Y_{0}\right)^2 \gg
%     \frac{Y_{0}}{\sqrt\rho \ROn}
% \end{equation*}
%
% Now, considering \(R = \ROn\) since it's the worst case scenario and
% returning to the use of impedance instead of admittance, the condition turns to
% \begin{equation*}
% \label{eq:ProtoOptParallelContrApprox}
%     \left(\frac{1}{\sqrt\rho} + 1\right)^2 \gg
%     \frac{Z_{0}}{\sqrt\rho \ROn}
% \end{equation*}
%
% Finally, we can use the worst possible value of \(\rho\) on each side
% (\(\rho = 1\) for the right side and \(\pi = \infty\) for the left) and
% arrive at
% \begin{equation}
% \label{eq:OptParallelContrApprox}
%     \ROn \gg Z_{0}
% \end{equation}

% \begin{frame}{Optimized contrast}
% \begin{gather*}
%     (RS^2 + Y_{0})^2 \gg S^2\\
%     \Downarrow\\
%     \left(\frac{R Y_{0}}{\sqrt\rho \ROn} + Y_{0}\right)^2 \gg
%     \frac{Y_{0}}{\sqrt\rho \ROn}\\
%     \Downarrow\\
%     \left(\frac{1}{\sqrt\rho} + 1\right)^2 \gg
%     \frac{Z_{0}}{\sqrt\rho \ROn}\\
%     \Downarrow\\
%     \ROn \gg Z_{0}
% \end{gather*}
% \end{frame}

%
% Now that we have finished our calculations, it is time to introduce the
% parasitic resistance. This is important because it depends on the design of
% the circuit, and plotting \(|\Delta\Gamma|(R_{p})\) will tell us with want
% amount of losses we can get our results.
%
% To introduce it we will give it the same treatment to \(R_{p}\) as to
% \(\ROff\) by introducing a ratio parameter
% \begin{equation*}
% \label{eq:PiDef}
%     \pi = \frac{R_{p}}{\ROn^{SET}}
% \end{equation*}

\begin{frame}{Introducing the parasitic resistance}
\begin{gather*}
    \pi = \frac{R_{p}}{\ROn^{SET}} \text{ with } 0 < \pi < \infty\\
    \\
\begin{split}
    R_{\text{On}}
    &= \frac{\pi}{1 + \pi} \ROn^{SET}\\
    \\
    R_{\text{Off}}
    &= \frac{\rho_{SET}\pi}{\rho_{SET} + \pi}\ROn^{SET}\\
    \\
    \rho
    &= \frac{\rho_{SET}(1 + \pi)}{\rho_{SET} + \pi}
\end{split}
\end{gather*}
\end{frame}

%
% With the important distinction that, as opposed to \(\rho\), \(0\leq\pi\leq\infty\),
% with \(\pi = 0\) being a short circuit and \(\pi = \infty\)
% being a lossless circuit.
%
% The substituions are then
%
% \begin{align*}
%     R_{\text{On}}
%     &= \frac{\pi}{1 + \pi} \ROn^{SET}\\
%     R_{\text{Off}}
%     &= \frac{\rho_{SET}\pi}{\rho_{SET} + \pi}\ROn^{SET}\\
%     \rho
%     &= \frac{\rho_{SET}(1 + \pi)}{\rho_{SET} + \pi}
% \end{align*}

% \begin{frame}{Introducing the parasitic resistance}
% \begin{align*}
%     R_{\text{On}}
%     &= \frac{\pi}{1 + \pi} \ROn^{SET}\\
%     R_{\text{Off}}
%     &= \frac{\rho_{SET}\pi}{\rho_{SET} + \pi}\ROn^{SET}\\
%     \rho
%     &= \frac{\rho_{SET}(1 + \pi)}{\rho_{SET} + \pi}
% \end{align*}
% \end{frame}

%
% With \(\rho_{SET} = \ROff^{SET}/\ROn^{SET}\). Taking this even further beyond
% with the fact that in an SET \(\rho_{SET} \approx \infty\)
% (in the off state, no electrons are travelling through), the substitutions are
%
% \begin{align*}
%     \ROn
%     &= \frac{\pi}{1 + \pi} \ROn^{SET}\\
%     \ROff
%     &= \pi \ROn^{SET}\\
%     \rho
%     &= 1 + \pi
% \end{align*}
%

\begin{frame}{Introducing the parasitic resistance}
With \(\rho_{SET} \approx \infty\)
\begin{align*}
    \ROn
    &= \frac{\pi}{1 + \pi} \ROn^{SET}\\
    \\
    \ROff
    &= \pi \ROn^{SET}\\
    \\
    \rho
    &= 1 + \pi\\
    \\
    |\Delta\Gamma| &\approx
                   2\left|
                   \frac{1 - \sqrt{1 + \pi}}{1 + \sqrt{1 + \pi}}
                   \right|\\
\end{align*}
\end{frame}

% The introduction of the parasitic resistance and an infinite \(\ROff^{SET}\)
% doesn't change much, in the sense that for most of the expressions is better
% to simply use \(\rho\) and \(\ROn\) for clarity. Most. Because for two results
% in specific it helps: in \ref{eq:OptimumParallelContr}
%
% \begin{equation}
% \label{eq:OptimumParallelContrOfPi}
%     |\Delta\Gamma| \approx
%                    2\left|
%                    \frac{1 - \sqrt{1 + \pi}}{1 + \sqrt{1 + \pi}}
%                    \right|
% \end{equation}
%
% And in \ref{eq:OptParallelContrApprox}
%
% \begin{equation}
% \label{eq:OptParallelContrApproxOfPi}
% \frac{\pi}{1 + \pi}\ROn^{SET} \gg Z_{0} \rightarrow
% \pi \gg \frac{Z_{0}}{\ROn^{SET} - Z_{0}}
% \end{equation}

% \begin{frame}{Introducing the parasitic resistance}
% \begin{gather*}
%     |\Delta\Gamma| \approx
%                    2\left|
%                    \frac{1 - \sqrt{1 + \pi}}{1 + \sqrt{1 + \pi}}
%                    \right|\\
%     % \frac{\pi}{1 + \pi}\ROn^{SET} \gg Z_{0} \rightarrow
%     % \pi \gg \frac{Z_{0}}{\ROn^{SET} - Z_{0}}\\
%     % \pi \gg \frac{50\unit{\ohm}}{50\unit{\kilo\ohm} + 50\unit{\ohm}} \approx 10^{-3}\\
%     % \\
%     % \text{If }\pi =  0.1\text{, }|\Delta\Gamma| \approx 0.0477
% \end{gather*}
% \end{frame}

%
% Using the values of \(Z_{0}\) and \(\ROn^{SET}\) that we have been considering
% up until now (\(50\unit{\ohm}\) and \(50\unit{\kilo\ohm}\) respectively) we can
% see that for \ref{eq:OptParallelContrApprox} to work in a worse case scenario,
% \(\pi\) must be a lot greater than \(10^{-3}\). It probably would
% be, given that for a \(\pi\) 100 times greater,
% \(|\Delta\Gamma| \approx 0.0477\), which isn't a good contrast to
% aim for.
%
% Finally, we check our results by comparing them against the numerically
% calculated optimum contrast via a simulation that searches the optimum
% value of \(C_{c}\) for a given value of \(\pi\).
%
% \begin{figure}[t]
% \centering
%   \includesvg[width=\linewidth]{RLCParallel/OptContrComparison.svg}
%   \caption{Numerical optimum contrast and our formula (\ref{eq:OptimumParallelContrOfPi})
%   with the difference between them.
% \(L = 180\unit{\nano\henry}\), \(C_{p} = 500\unit{\femto\farad}\),
% \(Z_{0} = 50\unit{\ohm}\), \(\ROn^{SET} = 50\unit{\kilo\ohm}\),
% \(\rho = 2\cdot10^{6}\). The script generating these images can be found
% in the companion Github repository with the name \texttt{OptContrComparison.py}}
% \label{fig:ParallelContrComparison}
% \end{figure}

\begin{frame}{Checking with a simulation}
\begin{table}[H]
    \centering
    \begin{tabular}{c|c|c|c|c}
        \(Z_{0}\) & \(\ROn^{SET}\) & \(\rho_{SET}\) & \(C_{p}\) & \(L\) \\\hline
        \(50\unit{\ohm}\) & \(50\unit{\kilo\ohm}\) & \(2\cdot10^{6}\) & \(500\unit{\femto\farad}\) & \(180\unit{\nano\henry}\)
    \end{tabular}
\end{table}
\end{frame}

\begin{frame}{Checking with simulations}
\begin{figure}[t]
\centering
  \includesvg[width=\linewidth]{RLCParallel/OptContrComparisonDefense.svg}
\end{figure}
\end{frame}

% As we can see in figure \ref{fig:ParallelContrComparison}, our
% estimations for what values of \(\pi\) were completely off, but in a good way.
% That is because even though the approximation gets worse for greater \(\pi\),
% it caps off at \(0.0175\), which is negligible in comparison to the values of
% \(|\Delta\Gamma|\) that we are aiming to.
%
% To summarize, we have found the expression for the value of \(C_{c}\) that
% optimizes the contrast. This value effectively tunes the resonator to the
% geometric mean of the 2 possible resistances of the circuit, and makes
% the contrast depend on only in their ratio. All of this was found with
% the constraints \ref{eq:ApproxLWr}, \ref{eq:ApproxCcWr} and
% \ref{eq:ApproxReflecCoeff}.

\end{document}
