\documentclass[../main.tex]{subfiles}

\begin{document}
\subsection{Kinetic inductance and his nonlinearity}


% \begin{frame}{Kinetic inductance nonlinearity}
% \begin{gather*}
%     E_{k} = \frac{1}{2}L_{k}I^2 = \frac{1}{2}m(nlS)v^2\\
%     \phantom{j=nqv} \Downarrow j=nqv\\
%     L_{k} = \frac{mlj^2}{nq^2Sj^2} = \frac{ml}{q^2S}\frac{1}{n}
% \end{gather*}
% \end{frame}


\begin{frame}{Kinetic inductance nonlinearity}
\begin{gather*}
    % L_{k} = \frac{ml}{q^2S}\frac{1}{n}\\
    % +\\
    % n(v) = |\Psi|^2 = \frac{1}{\beta}\left[|\alpha| - \frac{1}{2}m v^2\right]\\
    % \phantom{\text{ with } v \approx 0}\Downarrow\text{ with } v \approx 0\\
    L_{k} = L_{k0}\left(1 + \frac{j^2}{j_{*}^2} + \dots\right)\\
    % \text{With } L_{k0} = \frac{ml}{q^2 S n(v=0)}\text{, }
    % j_{*} = \frac{2p^2|\alpha|^{3}m}{\beta^2} = \sqrt{\frac{27}{4}}j_{c}\text{,}\\
    % \text{ and } j_c = \max j(v)
\end{gather*}
\end{frame}

% With \(L_{k0} = \frac{ml}{q^2Sn(v=0)}\) and \(j_{*}^2 =
% \frac{2q^2|\alpha|^3}{m\beta^2}\). If we compare \(j_{*}\)
% to the critical current \(j_{c}\), which is the maximum current that
% the system can withstand and can be calculated by obtaining the maximum of
% \(j\) with respect to \(v\), we get that
%
% \begin{equation*}
%     j_{*} = \sqrt{\frac{27}{4}}j_{c}
% \end{equation*}

% With this we can not only see that the kinetic inductance has a quadratic
% dependence with the current, but that the sensibility of that dependence it is
% given by the critical current of the material.
%
% But, the applications of inductors are usually with AC voltages, and in that
% case our kinetic inductance would be varying constantly. How can we use the
% kinetic inductor then?
%
% The solution is to introduce a DC bias to the circuit, with an intensity much
% greater than the maximum of the AC current, but still small enough to not break
% superconductivity and to use an AC current much smaller than the critical current.
% With it, we can have an inductor that changes inductance along with the resistance
% of the SET, since the effects of the AC current can be ignored.
%
% \begin{align*}
% \begin{split}
% L_{k} &= L_{k0}\left(1 + \frac{(j_{AC} + j_{DC})^2}{j_{*}^2} + \dots\right)\\
%       &= L_{k0}\left(1 + \frac{j_{DC}^2}{j_{*}^2} + \frac{j_{DC}j_{AC}}{j_{*}^2} + 
%       \frac{j_{AC}^2}{j_{*}^2} + \dots\right)\\
%       &= L_{k0}\left(1 + \frac{j_{DC}^2}{j_{*}^2} + (\frac{j_{DC}}{j_{*}} + 
%       \frac{j_{AC}}{j_{*}})\frac{j_{AC}}{j_{*}} + \dots\right)\\
%       &= L_{k0}\left(1 + \frac{j_{DC}^2}{j_{*}^2} + \dots\right)
%       \text{ since \(j_{DC} < j_{c}\) and \(j_{AC} \ll j_{c}\)}
% \end{split}
% \end{align*}

% \begin{frame}{Kinetic inductance nonlinearity}
% \begin{align*}
% L_{k} &= L_{k0}\left(1 + \frac{(j_{AC} + j_{DC})^2}{j_{*}^2} + \dots\right)\\
%       &= L_{k0}\left(1 + \frac{j_{DC}^2}{j_{*}^2} + \frac{j_{DC}j_{AC}}{j_{*}^2} + 
%       \frac{j_{AC}^2}{j_{*}^2} + \dots\right)\\
%       &= L_{k0}\left(1 + \frac{j_{DC}^2}{j_{*}^2} + (\frac{j_{DC}}{j_{*}} + 
%       \frac{j_{AC}}{j_{*}})\frac{j_{AC}}{j_{*}} + \dots\right)\\
%       &= L_{k0}\left(1 + \frac{j_{DC}^2}{j_{*}^2} + \dots\right)
%       \text{ since \(j_{DC} < j_{c}\) and \(j_{AC} \ll j_{c}\)}
% \end{align*}
% \end{frame}

\end{document}
